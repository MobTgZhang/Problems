\documentclass[UTF8,a4paper,10pt]{ctexart}
\usepackage[left=2.50cm, right=2.50cm, top=2.50cm, bottom=2.50cm]{geometry}

% -- text font --
% compile using Xelatex

%\setmainfont{Microsoft YaHei}  % 微软雅黑
%\setmainfont{YouYuan}  % 幼圆    
%\setmainfont{NSimSun}  % 新宋体
%\setmainfont{KaiTi}    % 楷体
%\setmainfont{SimSun}   % 宋体
%\setmainfont{SimHei}   % 黑体

\usepackage{times}
%\usepackage{mathpazo}
%\usepackage{fourier}
%\usepackage{charter}
%\usepackage{helvet}

\usepackage{amsmath, amsfonts, amssymb} % math equations, symbols
\usepackage[english]{babel}
\usepackage{color}      % color content
\usepackage{graphicx}   % import figures
\usepackage{url}        % hyperlinks
\usepackage{bm}         % bold type for equations
\usepackage{multirow}
\usepackage{booktabs}

\usepackage{fancyhdr}   % 设置页眉、页脚
%\pagestyle{fancy}
\lhead{}
\chead{}
\lfoot{}
\cfoot{}
\rfoot{}
\usepackage{hyperref}   % bookmarks
\hypersetup{colorlinks, bookmarks, unicode} % unicode


\title{每日数学练习}
\author{ MobtgZhang \thanks{A programmer, who is devoting to python,go,c++,and author's e-mail is mobtgzhang@outlook.com} }
\date{\today}

\begin{document}
	\maketitle
	\thispagestyle{fancy}
	\begin{flushleft}
		\begin{enumerate}
			\item 关键内容:stolz定理.\\
				由于$x_{n+1}=x_{n}(1-x_{n})$,从而我们可以得到$x_{n+1}-x_{n}=-x_{n}^{2}<0$.\\
				可见对于数列${x_{k}}_{k=1}^{n}$是单调递减数列,所以我们可以得到$0<x_{n}<1$.\\
				设$\lim\limits_{n\rightarrow\infty}x_{n}=a$,则可以得到等式$a=a(1-a)$,从而得到$a=0,\lim\limits_{n\rightarrow\infty}x_{n}=0$.\\
				所以由stolz定理可以得到
				\begin{equation}
					\lim\limits_{n\rightarrow\infty}\frac{1}{nx_{n}}=\lim\limits_{n\rightarrow\infty}(\frac{1}{x_{n}-\frac{1}{x_{n-1}}})
					=\lim\limits_{n\rightarrow\infty}\frac{x_{n-1}-x_{n}}{x_{n-1}x_{n}}=\lim\limits_{n\rightarrow\infty}\frac{1}{1-x_{n-1}}=1\nonumber
				\end{equation}
				所以得到$\lim\limits_{n\rightarrow\infty}nx_{n}=1$
			\item 关键内容:单调有界数列收敛定理、stolz定理.\\
			对于问题(a) 由数学归纳法可以得到$0\leq{x_{n}}\leq{1}$,由$x_{n+1}=\sin{x_{n}}<x_{n}$可以得到该数列是单调递减数列。由单调有界数列收敛定理可知$\lim\limits_{n\rightarrow\infty}x_{n}=0$\\
			对于问题(b),只需要证明$\lim\limits_{n\rightarrow\infty}\frac{1}{n\cdot{x_{n}^{2}}}=\frac{1}{3}$即可.\\
			由stolz定理可得
			\begin{equation}
				\lim\limits_{n\rightarrow\infty}\frac{1}{n\cdot{x_{n}^{2}}}=\lim\limits_{n\rightarrow\infty}(\frac{1}{x_{n}^{2}}-\frac{1}{x_{n-1}^{2}})
				=\lim\limits_{n\rightarrow\infty}\frac{x_{n-1}^{2}-\sin^{2}x_{n-1}^{2}}{x_{n-1}^{2}\sin^{2}x_{n-1}^{2}}
				=\lim\limits_{x\rightarrow{0}}\frac{x^{2}-\sin^{2}x}{x^{4}}=\frac{1}{3}\nonumber
			\end{equation}
			即证明了结论.
			\item 关键内容:单调有界数列收敛定理\\
			容易知道$\ln{(x+1)}<x$,故而当$x_{1}>0$时候,由于$x_{n+1}=\ln(1+x_{n})<x_{n}$,则数列${x_{k=1}}_{k=1}^{n}$是单调递减的,并且以0为下界.故而设极限$\lim\limits_{n\rightarrow\infty}x_{n}=a$,则得到等式$a=\ln(a+1)$,可以得到$a=0$.\\
			所以最后计算的极限:
			\begin{equation}
				\lim\limits_{n\rightarrow\infty}nx_{n}=\lim\limits_{n\rightarrow\infty}\frac{n}{\frac{1}{x_{n}}}=\lim\limits_{n\rightarrow\infty}
				=\lim\limits_{n\rightarrow\infty}\frac{n+1-n}{\frac{1}{x_{n+1}-\frac{1}{x_{n}}}}=\lim\limits_{n\rightarrow\infty}\frac{x_{n}x_{n+1}}{x_{n}-x_{n+1}}
				=\lim\limits_{n\rightarrow\infty}\frac{x_{n}\ln(x_{n}+1)}{x_{n}-\ln(x_{n}+1)}=2\nonumber
			\end{equation}
			从而可以得到$x_{n}\sim\frac{2}{n}(n\rightarrow+\infty)$
			\item 关键内容:导数的定义公式,有限增量公式\\
			设$\lambda=f{'}(x_{0})$,由有限增量公式可以得到$f(x_{0}+\Delta{x})=f(x_{0})+\lambda\Delta{x}+o(\Delta{x})$,于是可以得到
			\begin{equation}
				f(b_{n})=f(x_{0})+\lambda(b_{n}-x_{0})+o(b_{n}-x_{0})\nonumber
			\end{equation}
			\begin{equation}
				f(a_{n})=f(x_{0})+\lambda(a_{n}-x_{0})+o(a_{n}-x_{0})\nonumber
			\end{equation}
			这样,我们考察另一个无穷小变量:
			\begin{equation}
				\frac{o(b_{n}-x_{0})+o(a_{n}-x_{0})}{b_{n}-a_{n}}=\frac{o(b_{n}-x_{0})}{b_{n}-x_{0}}\cdot{\frac{o(b_{n}-x_{0})}{b_{n}-a_{n}}}
				+\frac{o(a_{n}-x_{0})}{a_{n}-x_{0}}\cdot{\frac{o(a_{n}-x_{0})}{b_{n}-a_{n}}}\nonumber
			\end{equation}
			由于$0<\frac{b_{n}-x_{0}}{b_{n}-a_{n}}<1,-1<\frac{a_{n}-x_{0}}{b_{n}-a_{n}}<0$有界,则得到
			\begin{equation}
				\lim\limits_{n\rightarrow\infty}\frac{o(b_{n}-x_{0})+o(a_{n}-x_{0})}{b_{n}-a_{n}}=0\nonumber
			\end{equation}
			即
			\begin{equation}
				o(b_{n}-x_{0})+o(a_{n}-x_{0}) = o(b_{n}-a_{0})\nonumber
			\end{equation}
			这时候有
			\begin{equation}
				f(b_{n})-f(a_{n})=\lambda(b_{n}-a_{n})+o(b_{n}-x_{0})+o(a_{n}-x_{0}) =\lambda(b_{n}-a_{n})+ o(b_{n}-a_{0})\nonumber
			\end{equation}
			从而由导数的定义可以得到
			\begin{equation}
				\lim\limits_{n\rightarrow\infty}\frac{f(b_{n})-f(a_{n})}{b_{n}-a_{n}}=f{'}(x_{0})\nonumber
			\end{equation}
			\item 关键内容:函数求导\\
			由于$u,v$的任意性质,所以可以得到$\frac{f(u)-f(v)}{u-v}=\alpha f{'}(u)+\beta f{'}(v)=\alpha f{'}(v)+\beta f{'}(u)$,从而可以得到$(f{'}(u)-f{'}(v))(\alpha-\beta)=0$.所以可以得到$\alpha=\beta=\frac{1}{2}$或者$f{'}(x)=C$.后者可以得到$f(x)=Cx+C_{1}$.对于前者,带入表达式可以得到
			\begin{equation}
				2(f(u)-f(v))=(u-v)(\alpha f{'}(u)+\beta f{'}(v))\nonumber
			\end{equation}
			对上述方程对$u$进行求导,可以得到
			\begin{equation}
				2f{'}(u)=\alpha f{'}(u)+\beta f{'}(v)+(u-v)\alpha f{''}(u)\nonumber
			\end{equation}
			整理之后得到
			\begin{equation}
				f{'}(u)=\beta f{'}(v)+(u-v)\alpha f{''}(u)\nonumber
			\end{equation}
			继续对$u$进行求导,得$f{''}=0$,所以得到$f(x)=ax^{2}+bx+c$
			综上所述,所以得到$f(x)$的表达式为$f(x)=ax^{2}+bx+c$
			\item 关键内容:导数求导方法\\
			先证明其充分性\\
			作辅助函数
			\begin{equation}
				f(t)=\frac{F(tx,ty)}{t^{k}} \nonumber
			\end{equation}
			则得到
			\begin{equation}
				f{'}(t)=\frac{txF_{1}{'}(tx,ty)+tyF_{2}{'}(tx,ty)-kF(tx,ty)}{t^{k+1}}=0\nonumber
			\end{equation}
			可见$f(t)$是一个常数函数.特别地,$f(1)=F(x,y)$,故而可以得到$F(tx,ty)=t^{k}F(x,y)$\\
			再证明其必要性\\
			对表达式$F(tx,ty)=t^{k}F(x,y)$两边对$t$求导可以得到
			\begin{equation}
				xF{'}_{1}(tx,ty)+yF{'}_{2}(tx,ty)=kt^{k-1}F(x,y)\nonumber
			\end{equation}
			令$t=1$,则可以得到$xF{'}_{1}(x,y)+yF{'}_{2}(x,y)=kF(x,y)$
		\end{enumerate}
	\end{flushleft}
\end{document}
