\documentclass[UTF8,a4paper,10pt]{ctexart}
\usepackage[left=2.50cm, right=2.50cm, top=2.50cm, bottom=2.50cm]{geometry}

% -- text font --
% compile using Xelatex

%\setmainfont{Microsoft YaHei}  % 微软雅黑
%\setmainfont{YouYuan}  % 幼圆    
%\setmainfont{NSimSun}  % 新宋体
%\setmainfont{KaiTi}    % 楷体
%\setmainfont{SimSun}   % 宋体
%\setmainfont{SimHei}   % 黑体

\usepackage{times}
%\usepackage{mathpazo}
%\usepackage{fourier}
%\usepackage{charter}
%\usepackage{helvet}

\usepackage{amsmath, amsfonts, amssymb} % math equations, symbols
\usepackage[english]{babel}
\usepackage{color}      % color content
\usepackage{graphicx}   % import figures
\usepackage{url}        % hyperlinks
\usepackage{bm}         % bold type for equations
\usepackage{multirow}
\usepackage{booktabs}

\usepackage{fancyhdr}   % 设置页眉、页脚
%\pagestyle{fancy}
\lhead{}
\chead{}
\lfoot{}
\cfoot{}
\rfoot{}
\usepackage{hyperref}   % bookmarks
\hypersetup{colorlinks, bookmarks, unicode} % unicode


\title{每日数学练习}
\author{ MobtgZhang \thanks{A programmer, who is devoting to python,go,c++,and author's e-mail is mobtgzhang@outlook.com} }
\date{\today}

\begin{document}
	\maketitle
	\thispagestyle{fancy}
	\begin{flushleft}
		\begin{enumerate}
			\item 设
				\begin{equation}
				x_{1}\in(0,1),x_{n+1}=x_{n}(1-x_{n})(n=1,2,...) \nonumber
				\end{equation}
				证明:
				\begin{equation}
				\lim\limits_{n\rightarrow\infty}nx_{n}=1\nonumber
				\end{equation}
			\item 设$x_{1}=\sin x_{0}>0,x_{n+1}=\sin x_{n}(n=1,2,3,...)$,证明以下的结论:
				\begin{enumerate}
					\item $\lim\limits_{n\rightarrow\infty}x_{n}=0$
					\item $\lim\limits_{n\rightarrow\infty}x_{n}\sqrt{\frac{n}{3}}=1$
				\end{enumerate}
			\item 设$x_{1}>0,x_{n+1}=\ln(1+x_{n})(n=1,2,3,...)$,证明
				\begin{equation}
					\lim\limits_{n\rightarrow\infty}x_{n}=0\nonumber
				\end{equation}
				并且$x_{n}\sim\frac{2}{n}(n\rightarrow+\infty)$
			\item 设$f(x)$在点$x_{0}$处可导,$a_{n}<x_{0}<b_{n}$并且有$\lim\limits_{n\rightarrow\infty}a_{n}=\lim\limits_{n\rightarrow\infty}b_{n}=x_{0}$,证明:
			\begin{equation}
				\lim\limits_{n\rightarrow\infty}\frac{f(b_{n})-f(a_{n})}{b_{n}-a_{n}}=f{'}(x_{0})\nonumber
			\end{equation}
			\item 若可导函数$f(x)$对于一切$u,v$恒有
			\begin{equation}
				\frac{f(u)-f(v)}{u-v}=\alpha f{'}(u)+\beta f{'}(v)\nonumber
			\end{equation}
			成立,其中常数$\alpha,\beta>0$并且$\alpha+\beta=1$.求$f(x)$的表达式
			\item 对于函数$F(x,y)$,如果存在常数$k$,使得对于任何$x,y$以及$t>0$恒有$F(tx,ty)=t^{k}F(x,y)$成立,则称$F(x,y)$是$k$次齐次多项式.证明:可微函数$F(x,y)$是$k$次齐次函数的充要条件为对于任何$x,y$恒有$xF_{1}{'}(x,y)+yF_{2}{'}(x,y)=kF(x,y)$成立.
		\end{enumerate}
	\end{flushleft}
\end{document}
