\documentclass[UTF8,a4paper,10pt]{ctexart}
\usepackage[left=2.50cm, right=2.50cm, top=2.50cm, bottom=2.50cm]{geometry}
\usepackage{times}
\usepackage{amsmath, amsfonts, amssymb} % math equations, symbols
\usepackage[english]{babel}
\usepackage{color}      % color content
\usepackage{graphicx}   % import figures
\usepackage{url}        % hyperlinks
\usepackage{bm}         % bold type for equations
\usepackage{multirow}
\usepackage{booktabs}
\usepackage{fancyhdr}   % 设置页眉、页脚
\lhead{}
\chead{}
\lfoot{}
\cfoot{}
\rfoot{}
\usepackage{hyperref}   % bookmarks
\hypersetup{colorlinks, unicode} % unicode
\title{每日数学练习}
\author{ MobtgZhang \thanks{A programmer, who is devoting to python,go,c++,and author's e-mail is mobtgzhang@outlook.com} }
\date{\today}

\begin{document}
	\maketitle
	\thispagestyle{fancy}
	\begin{flushleft}
		\begin{enumerate}
			\item 关键内容:因式分解\\
			\begin{enumerate}
				\item 由于 
				\begin{equation}
					\frac{1}{1+x^{4}}=\frac{1}{1+x^{4}+2x^{2}-2x^{2}}\nonumber
				\end{equation}
				\begin{equation}
					=\frac{1}{(x^{2}+\sqrt{2}x+1)(x^{2}-\sqrt{2}x+1)}
					=\frac{1}{2\sqrt{2}}(\frac{x+\sqrt{2}}{x^{2}+\sqrt{2}x+1}+\frac{-x+\sqrt{2}}{x^{2}-\sqrt{2}x+1})\nonumber
				\end{equation}
				\begin{equation}
					=\frac{1}{2\sqrt{2}}(\frac{x+\frac{\sqrt{2}}{2}}{x^{2}+\sqrt{2}x+1}+\frac{-x+\frac{\sqrt{2}}{2}}{x^{2}-\sqrt{2}x+1})
					+\frac{1}{4}(\frac{1}{x^{2}+\sqrt{2}x+1}+\frac{1}{x^{2}-\sqrt{2}x+1})\nonumber
				\end{equation}
				其中后者可以变换为
				\begin{equation}
					\frac{1}{x^{2}+\sqrt{2}x+1}+\frac{1}{x^{2}-\sqrt{2}x+1}=\frac{1}{(x+\frac{\sqrt{2}}{2})^{2}+\frac{1}{2}}+\frac{1}{(x-\frac{\sqrt{2}}{2})^{2}+\frac{1}{2}}
				\end{equation}
				所以前者的积分为
				\begin{equation}
					\frac{1}{4\sqrt{2}}\ln{\frac{x^{2}+\sqrt{2}x+1}{x^{2}-\sqrt{2}x+1}}\nonumber
				\end{equation}
				后者的积分为
				\begin{equation}
					\frac{1}{2\sqrt{2}}\arctan{\frac{\sqrt{2}x}{1-x^{2}}}\nonumber
				\end{equation}
				所以最后的积分为
				\begin{equation}
					\int\frac{1}{1+x^{4}}dx=\frac{1}{4\sqrt{2}}\ln{\frac{x^{2}+\sqrt{2}x+1}{x^{2}-\sqrt{2}x+1}}+\frac{1}{2\sqrt{2}}\arctan{\frac{\sqrt{2}x}{1-x^{2}}}+C
				\end{equation}
				\item
				\begin{equation}
					\int\frac{dx}{x^{6}(x+1)}=\int{\frac{-x^{5}+x^{4}-x^{3}+x^{2}-x+1}{x^{6}}+\frac{1}{x+1}}dx
					=\ln{|\frac{x+1}{x}|}-\frac{1}{x}+\frac{1}{2x^{2}}-\frac{1}{3x^{3}}+\frac{1}{4x^{4}}-\frac{1}{5x^{5}}+C\nonumber
				\end{equation}
			\end{enumerate}
			\item 关键内容:原函数的定义等等\\
				由题设可知$\int_{0}^{x}{\frac{\sin{t}}{t}dx=f(x)}$而
				\begin{equation}
					\int{xf{'}(2x)}dx=\frac{1}{2}\int{x}df(2x)=\frac{1}{2}xf(2x)-\frac{1}{2}\int{f(2x)}dx\nonumber
				\end{equation}
				\begin{equation}
					=\frac{1}{2}xf(2x)-\frac{1}{4}\int{f(2x)}d(2x)=\frac{1}{4}\cos{2x}-\frac{1}{4x}\sin{2x}+C\nonumber
				\end{equation}
			\item 关键内容:分段函数的求法\\
				当$x<0$时候,$\int{f(x)}dx=\int{1}dx=x+C_{1}$.\\
				当$0\leq{x}\leq{1}$时候,$\int{f(x)}dx=\int{x+1}dx=\frac{1}{2}x^{2}+x+C_{2}$.\\
				当$x>1$时候,$\int{f(x)}dx=\int{2x}dx=x^{2}+C_{3}$\\
				考虑到函数的连续性,即在点$x=0,x=1$处积分函数是连续的,所以有$C_{1}=C_{2},\frac{1}{2}+1+C_{2}=1+C_{3}$
				从而得到$C_{2}=C_{1},C_{3}=\frac{1}{2}+C_{1}$
				所以得到
				\begin{equation}
					\int{f(x)}dx=\begin{cases}
						x+C&,x<0\\
						\frac{1}{2}x^{2}+x+C&,0\leq{x}\leq{1}\\
						x^{2}+\frac{1}{2}+C&,x>1\\
					\end{cases}\nonumber
				\end{equation}
			\item 关键内容:莱布尼兹求导公式\\
			由题目中的表述可以得到$(1-x-x^{2})f(x)=1$.\\
			对上式求$n$阶导数可以得到
			\begin{equation}
			(1-x-x^{2})f^{(n)}(x)-n(1+2x)f^{(n-1)}(x)-n(n-1)f^{(n-2)}(x)=0\nonumber
			\end{equation}
			令$x=0$可以得到
			\begin{equation}
			f^{(n)}(0)-nf^{(n-1)}(0)-n(n-1)f^{(n-2)}(0)=0\nonumber
			\end{equation}
			所以得到递推公式
			\begin{equation}
			\frac{f^{(n)}(0)}{n!}-\frac{nf^{(n-1)}(0)}{n!}-\frac{n(n-1)f^{(n-2)}(0)}{n!}\nonumber
			\end{equation}
			即$a_{n}=a_{n-1}+a_{n-2}$
			容易得到
			\begin{equation}
				\sum\limits_{n=0}^{\infty}\frac{a_{n+1}}{a_{n}a_{n+2}}=\sum\limits_{n=0}^{\infty}\frac{a_{n+2}-a_{1}}{a_{n}a_{n+2}}=\sum\limits_{n=0}^{\infty}(\frac{1}{a_{n}}-\frac{1}{a_{n+2}})=\lim\limits_{b\rightarrow\infty}(\frac{1}{a_{0}}+\frac{1}{a_{1}}-\frac{1}{a_{n+1}}-\frac{1}{a_{n+2}})\nonumber
			\end{equation}
			由于$a_{0}=1,a_{1}=1$,所以由数学归纳法可以得到$a_{n}>n$,那么$\lim\limits_{n\rightarrow\infty}\frac{1}{a_{n}}=0$
			所以最后得到$\sum\limits_{n=0}^{\infty}\frac{a_{n+1}}{a_{n}a_{n+2}}=2$
			\item 关键内容:数列的递推与计算、数学归纳法\\
			下面用数学归纳法证明,数列的部分和有:
			\begin{equation}
				S_{n}=\sum\limits_{k=1}^{n}\frac{1}{u_{k}}=1-\frac{1}{u_{n+1}-1}\nonumber
			\end{equation}
			当$n=1$的时候,$S_{1}=\frac{1}{u_{1}}=\frac{1}{2}$显然成立.\\
			假设当$n=m$的时候,
			\begin{equation}
				S_{m}=\sum\limits_{k=1}^{m}\frac{1}{u_{k}}=1-\frac{1}{u_{m+1}-1}\nonumber
			\end{equation}
			成立.
			当$n=m+1$的时候,
			\begin{equation}
				S_{m}=\sum\limits_{k=1}^{m}\frac{1}{u_{k}}=1-\frac{1}{u_{m+1}-1}+\frac{1}{u_{m+1}}=1-\frac{1}{u_{m+1}(u_{m+1}-1)}=1-\frac{1}{u_{m+2}-1}\nonumber
			\end{equation}
			即对任意正整数$n$上式均成立.所以证明了结论.
			容易得到$u_{n+1}-u_{n}=(u_{n}+1)^{2}$即数列是单调增加的,并且有$u_{n}\geq{2}$.而
			\begin{equation}
				u_{n}-1=u_{n-1}(u_{n-1}-1)=...=u_{n-1}u_{n-2}...u_{2}(u_{1}-1)\geq{2^{n-1}}
			\end{equation}
			显然$\lim\limits_{n\rightarrow\infty}u_{n}=\infty$,所以得到$\sum\limits_{k=1}^{n}\frac{1}{u_{k}}=1$
			\item 关键内容:级数表达式的变形、Stolz定理\\
			设
			\begin{equation}
				a_{k}=1+\frac{1}{2}+...+\frac{1}{k}\nonumber
			\end{equation}
			那么级数的部分和$S_{n}$可以写作
				\begin{equation}
					S_{n}=\sum\limits_{k=1}^{n}\frac{a_{k}}{(k+1)(k+2)}=\sum\limits_{k=1}^{n}(\frac{a_{k}}{k+1}-\frac{a_{k}}{k+2})\nonumber
				\end{equation}
				\begin{equation}
					=(\frac{a_{1}}{2}-\frac{a_{1}}{3})+(\frac{a_{2}}{3}-\frac{a_{2}}{4})+...+(\frac{a_{n-1}}{n}-\frac{a_{n-1}}{n+1})+(\frac{a_{n}}{n+1}-\frac{a_{n}}{n+2})\nonumber
				\end{equation}
				\begin{equation}
					=\frac{a_{1}}{2}+\frac{a_{2}-a_{1}}{3}+\frac{a_{3}-a_{2}}{4}+...+\frac{a_{n}-a_{n-1}}{n+1}-\frac{a_{n}}{n+2}\nonumber
				\end{equation}
				\begin{equation}
					=\frac{1}{2\times{1}}+\frac{1}{3\times{2}}+\frac{1}{4\times{3}}+...+\frac{1}{(n+1)\times{n}}-\frac{a_{n}}{n+2}\nonumber
				\end{equation}
				\begin{equation}
					=1-\frac{1}{n+1}-\frac{a_{n}}{n+2}\nonumber
				\end{equation}
				由Stolz定理可得$\lim\limits_{n\rightarrow\infty}\frac{a_{n}}{n+2}=0$,则可以得到$\lim\limits_{n\rightarrow\infty}S_{n}=1$
		\end{enumerate}
	\end{flushleft}
\end{document}
